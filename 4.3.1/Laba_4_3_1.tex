\documentclass[a4paper, 12pt]{article}%тип документа

%отступы
\usepackage[left=0.6cm,right=1cm,top=2cm,bottom=3cm,bindingoffset=0cm]{geometry}
\setlength{\parindent}{5ex}

%Русский язык
\usepackage[T2A]{fontenc} %кодировка
\usepackage[utf8]{inputenc} %кодировка исходного кода
\usepackage[english,russian]{babel} %локализация и переносы

%Вставка картинок
\usepackage{graphicx}
\graphicspath{{pictures/}}
\DeclareGraphicsExtensions{.pdf,.png,.jpg}

%Графики
\usepackage{pgfplots}
\pgfplotsset{compat=1.9}

%Математика
\usepackage{amsmath, amsfonts, amssymb, amsthm, mathtools}

%Таблицы
\usepackage{longtable} 
\usepackage{float}

%Римские цифры
\newcommand{\RomanNumeralCaps}[1]{\uppercase\expandafter{\romannumeral#1}}

\usepackage{multirow}


\begin{document}
	\begin{titlepage}
		\begin{center}
			\textsc{Федеральное государственное автономное образовательное учреждение высшего образования«Московский физико-технический институт (национальный исследовательский университет)»\\[5mm]
			}
			
			\vfill
			
			\textbf{Отчёт по лабораторной работы 4.3.1 \\[3mm]
				ИЗУЧЕНИЕ ДИФРАКЦИИ СВЕТА
				\\[50mm]
			}
			
		\end{center}
		
		\hfill
		\begin{minipage}{.5\textwidth}
			Выполнил студент:\\[2mm]
			Сериков Алексей Романович\\[2mm]
			группа: Б03-103\\[5mm]
			
		\end{minipage}
		\vfill
		\begin{center}
			Москва, 2023 г.
		\end{center}
		
	\end{titlepage}
	
	\newpage
	\textbf{Аннотация}\\
	
	
	\textbf{Цель работы: }\\
	
	Исследовать явления дифракции Френеля и Фраунгофера на щели, изучить влияние дифракции на разрешающую способность оптических инструментов.\\
	
	\textbf{В работе используются: }\\
	
	Оптическая скамья, ртутная лампа, светофильтр, щели с регулируемой шириной, рамка с вертикальной нитью, двойная щель, микроскоп на поперечных салазках с микрометрическим винтом, зрительная труба.\\
	
\RomanNumeralCaps 1 \textit{Дифракция Френеля:}\\
	
	
	Схема установки для наблюдения дифракции Френеля на щели представлена на рис. \ref{LabA}. Световые лучи освещают щель $ S_2 $ и испытывают на ней дифракцию. Дифракционная картина рассматривается с помощью микроскопа М, сфокусированного на некоторую плоскость наблюдения П.
	
	\begin{figure}[H]
		\begin{center}
			\includegraphics[width=0.8\linewidth]{a.pdf}
			\caption{Схема установки для наблюдения дифракции Френеля}
			\label{LabA}
		\end{center}
	\end{figure}
	
	Щель $ S_2 $ освещается параллельным пучком монохроматического света с помощью коллиматора, образованного объективом $ O_1 $, и щелью $S_1$, находящейся в его фокусе. На щель $ S_1 $ сфокусировано изображение спектральной линии, выделенной из спектра ртутной лампы Л при помощи простого монохроматора C, в котором используется призма прямого зрения. Распределение интенсивности света в плоскости наблюдения П проще всего рассчитывать с помощью зон Френеля (для щели их иногда называют зонами Шустера). При освещении щели $ S_2 $ параллельным пучком лучей (плоская волна) зоны Френеля представляют собой полоски, параллельные краям щели (рис. \ref{zone}). Результирующая амплитуда в точке наблюдения определяется суперпозицией колебаний от тех зон Френеля, которые не перекрыты створками щели. Графическое определение результирующей амплитуды производится с помощью векторной диаграммы --- спирали Корню. Суммарная ширина $ n $ зон Френеля (Шустера) определяется соотношением:
	
	\begin{equation}\label{xin}
		\xi_n = \sqrt{zn\lambda}
	\end{equation}
	где $ z $ --- расстояние от щели до плоскости наблюдения рис.\ref{LabA}, а $ \lambda $ --- длина волны.
	
	\begin{figure}[H]
		\begin{center}
			\includegraphics[width=0.3\linewidth]{zone.pdf}
			\caption{Зоны Френеля}
			\label{zone}
		\end{center}
	\end{figure}

	
	Вид наблюдаемой дифракционной картины
	на щели шириной $ b $ определяется волновым параметром $ p $ или числом Френеля $ C $ (число открытых полных зон):
	
	
	\begin{equation}\label{}
		p = \dfrac{\sqrt{z \lambda}}{b}, \qquad C = \dfrac{1}{p^2}
	\end{equation}
	
	Дифракционная картина отсутствует вблизи щели при $ p \ll 1 $ ($ C \gg 1 $, т. е. на щели укладывается огромное число зон), а распределение интенсивности света за щелью можно приближённо получить с помощью законов геометрической оптики. Дифракционная картина в этом случае наблюдается только в узкой области на границе света и тени у краёв экрана.
	
	При небольшом удалении от щели (или изменении ширины щели $ S_2 $) эти две группы дифракционных полос перемещаются практически независимо друг от друга. Каждая из этих групп образует картину дифракции Френеля на краю экрана. Распределение интенсивности при дифракции света на краю экрана может быть найдено с помощью спирали Корню.
	
	При дальнейшем увеличении расстояния $ z $ (или уменьшении ширины щели $ S_2 $) обе системы дифракционных полос постепенно сближаются и, наконец, при $ C \gtrsim 1 $ накладываются друг на друга. Распределение интенсивности в плоскости наблюдения в этом случае определяется числом зон Френеля, укладывающихся на полуширине щели $ b/2 $. Если это число равно $ n $, то в поле зрения наблюдается $ m = n - 1 $ тёмных полос. Таким образом, по виду дифракционной картины можно оценить число зон Френеля на полуширине щели.   
	
	
		
	\textbf{Ход работы и обработка результатов.}\\
	
	\textit{Погрешности измерений:}\\
	Линейка: $\sigma = \pm 1$мм\\
	Микроскоп: $\sigma = \pm 0.01$мм
	Формула погрешности произведения и частного:\\
	\begin{equation}
		\frac{\sigma_u}{u}=\sqrt{\left(\frac{\sigma_x}{x}\right)^2+\left(\frac{\sigma_y}{y}\right)^2}
	\end{equation}
	Формула погрешности разности и суммы:
	\begin{equation}
		\sigma_u=\sqrt{\sigma_x^2+\sigma_y^2}
	\end{equation}
	
	Приближая микроскоп к щели, измерим зависимость координаты микроскопа $z$ от числа $n$ наблюдаемых тёмных полос:
	
	$x_0$ = 67.3 cм - начальное положение микроскопа.
	
	$z_n$ = $x_0$ - $x_n$
	
	
	Также по формуле (1) рассчитаем $\xi_n$, где $\lambda$ = $5461 \cdot 10^{-10}$
	
	
	\begin{longtable}{|c|c|c|c|c|c|c|}
		\hline
		n& 1&  2 & 3& 4 & 5& 6\\ \hline
		$z_n$, cм& 2.8& 2.6 & 1.9& 1.4 & 1.1& 0.9\\ \hline
		$\xi_n$, мм& 1.2& 1.7 & 1.76& 1.74 & 1.73& 1.71\\ \hline
		\caption{Таблица с данными зависимости координаты микроскопа $z$ от числа $n$ наблюдаемых тёмных полос.}
	\end{longtable}

Построим график (рис. \ref{1}) зависимости $\xi(m)$, где $m$ = $n$ + 1. убедимся в том, что
экспериментальные точки могут быть аппроксимированы прямой линией. По наклону наилучшей прямой определим ширину щели $b$. Оценим погрешность результата.

\begin{figure}[H]
	\begin{center}
		\includegraphics[width=0.8\linewidth]{1.pdf}
		\caption{Зависимость $2\xi(m)$}
		\label{1}
	\end{center}
\end{figure}

Измерим ширину $b$ щели $S_2$ с помощью микроскопа: $b$ = 0.39 мм.

Получим $b$ = $0.39 \pm 0.02$ мм и $b$ = $0.35 \pm 0.02$ мм (с помощью графика \ref{1}). Результаты совпадают в пределах погрешности.

\newpage
	\RomanNumeralCaps 2 \textit{Дифракция Фраунгофера:}\\
	
	На значительном удалении от щели, когда выполнено условие $ C \ll 1 $
	(то есть ширина щели становится значительно меньше ширины первой
	зоны Френеля, $ b \ll \sqrt{\lambda z} $), изображение щели размывается и возникает
	дифракционная картина, называемая дифракцией Фраунгофера.
	
	Дифракцию Френеля и Фраунгофера можно наблюдать на одной
	и той же установке (рис. \ref{LabA}). Однако при обычных размерах установки дифракция Фраунгофера возникает только при очень узких щелях.
	
		\begin{figure}[H]
		\begin{center}
			\includegraphics[width=0.8\linewidth]{b.pdf}
			\caption{Схема установки для наблюдения дифракции Фраунгофера}
			\label{LabB}
		\end{center}
	\end{figure}
	
	Например, при $ z \approx $ 20-40 см и $  \lambda \approx 5  $ 10-5  см получаем$  b \ll 0,3 $ мм. Поскольку работать с такими тонкими щелями неудобно, для наблюдения дифракции Фраунгофера к схеме, изображённой на рис. \ref{LabA}, добавляется объектив $ O_2  $ (рис. \ref{LabB}).
	
	Дифракционная картина наблюдается здесь в фокальной плоскости
	объектива $ O_2 $. Каждому значению угла $ \theta $ соответствует в этой плоскости точка, отстоящая от оптической оси на расстоянии
	
	\begin{equation}\label{x}
		x = f_2 \tg \theta \approx f_2 \theta
	\end{equation}
	
	Поскольку объектив не вносит дополнительной разности хода
	между интерферирующими лучами (таутохронизм), в его фокальной
	плоскости наблюдается неискаженная дифракционная картина Фраунгофера. Эта картина соответствует бесконечно удалённой плоскости
	наблюдения.
	
	В центре поля зрения наблюдается дифракционный максимум (светлая полоса). При малых углах $ \theta $ положение минимумов (тёмных полос)
	определяется, соотношением
	
	\begin{equation}\label{theta_m}
		\theta_m = m \dfrac{\lambda}{b}
	\end{equation}
	
	Расстояние $ x_m $ от тёмной полосы до оптической оси объектива $ O_2 $ пропорционально фокусному расстоянию $ f_2 $. Из \eqref{x} и \eqref{theta_m} следует 
	
	\begin{equation}\label{xm}
		x_m = m \dfrac{\lambda}{b} f_2
	\end{equation}
	
	Видно, что при малых углах минимумы эквидистантны, а расстояния $ \delta x $ между минимумами обратно пропорциональны ширине $ b $ щели $ S_2 $
	
	\textbf{Ход работы и обработка результатов.}\\
	
	Измерим с помощью окулярной шкалы микроскопа координаты $x_m$ 5 дифракционных минимумов в обе стороны
	от центра:
	
		\begin{longtable}{|c|c|c|c|c|c|c|c|c|c|c|}
		\hline
		m& -1&  -2 & -3& -4 & -5 & 1&  2 & 3& 4 & 5  \\ \hline
		$x_m$&0.1& 0.3 & 0.5& 0.7 & 0.9& 0.1& 0.3 & 0.5& 0.7 & 0.9\\ \hline
		\caption{Таблица с данными координат дифракционных минимумов.}
	\end{longtable}

Ширина щели $S_2$ $b$ = 0.4мм, фокусное расстояние $f_2$ = 12.5 см.


Построим график зависимости положений $x_m$ экстремумов дифракционной картины
от их номера $m$. Убедимся, что зависимость может быть аппроксимирована прямой
линией. По наклону прямой определим ширину щели $b$. Оценим погрешность результата.

\begin{figure}[H]
	\begin{center}
		\includegraphics[width=0.8\linewidth]{2.pdf}
		\caption{Зависимость $x(m)$}
		\label{2}
	\end{center}
\end{figure}
	
	По формуле (6) получим $b$ = $0.20 \pm 0.02$ мм и $b$ = $0.19 \pm 0.02$ мм (с помощью МНК из графика \ref{2}). Результаты совпадают в пределах погрешности.
	
	\newpage
\RomanNumeralCaps 3 \textit{Дифракция Фраунгофера на двух щелях:}\\
Для наблюдения дифракции Фраунгофера на двух щелях в установке (рис. \ref{LabB}) следует заменить щель $ S_2 $ экраном Э с двумя щелями
(рис. \ref{labC}). При этом для оценки влияния ширины входной щели на чёткость дифракционной картины вместо входной щели $ S_1 $ следует поставить щель с микрометрическим винтом. Два дифракционных изображения входной щели, одно из которых образовано лучами, прошедшими через левую, а другое --- через правую щели, накладываются друг на друга.

\begin{figure}[h!]
	\centering
	\includegraphics[width=0.8\linewidth]{c.pdf}
	\caption{Схема установки для наблюдения дифракции Фраунгофера на двух щелях}
	\label{labC}
\end{figure}

Если входная щель достаточно узка, то дифракционная картина
в плоскости П (рис. \ref{labC}) подобна той, что получалась при дифракции
на одной щели (рис. \ref{LabB}), однако теперь вся картина испещрена рядом
дополнительных узких полос.
Угловая координата $ \theta_m $ интерференционного максимума $ m $-го порядка определяется соотношением

\begin{equation}\label{}
	\theta_m = m \dfrac{\lambda}{b}
\end{equation}

где $ d $ --- расстояние между щелями. Линейное расстояние $ \delta x $ между соседними интерференционными полосами в плоскости П равно, поэтому

\begin{equation}\label{dx}
	\delta x = f_2 \dfrac{\lambda}{d}
\end{equation}

На рис. \ref{labC} показано распределение интенсивности в фокальной плоскости объектива $ O_2 $. Штриховой линией (в увеличенном масштабе)
изображено распределение интенсивности при дифракции света на одиночной щели. Нетрудно оценить число n интерференционных полос,
укладывающихся в области центрального дифракционного максимума.
Согласно \eqref{xm} полная ширина главного максимума равна $ 2 f_2 \lambda /b $, где $ b $ ширина щели, отсюда

\begin{equation}\label{n}
	n = \dfrac{2f_2 \lambda}{b} \dfrac{1}{\delta x} = \dfrac{2d}{b}
\end{equation}

При дифракции света на двух щелях чёткая система интерференционных полос наблюдается только при достаточно узкой ширине входной щели $ S $, которую можно рассматривать как протяжённый источник света размером $ b $. Для наблюдения интерференции необходимо, чтобы расстояние $ d $между щелями не превышало радиуса когерентности

\begin{equation}\label{}
	d \ll \dfrac{\lambda}{b} f_1
\end{equation}

Здесь $ b $ --- ширина входной щели $ S $ и, следовательно, $  b/f_1 $ --- её угловая ширина. Таким образом, по размытию интерференционной картины можно оценить размер источника. Этот метод используется в звёздном интерферометре при измерении угловых размеров звёзд.

\textbf{Ход работы и обработка результатов.}\\

По расстоянию $\delta x$ между полосами и по формуле (9), рассчитаем расстояние расстояние между щелями $d$ и сравним с измеренным.

$d$ = $0.65 \pm 0.05$ - по формуле (9)

$d$ = $0.71 \pm 0.01$ - по измерениям


Наблюдаемое число полос в главном максимуме $N = 7$, а по формуле (10) $N = 6.5$

\RomanNumeralCaps 4 \textit{Влияние дифракции на разрешающую способность оптического инструмента:}\\

Установка, представленная на рис. \ref{LabB}, позволяет исследовать влияние дифракции на разрешающую способность оптических инструментов.

Как уже было выяснено, линзы $O_1$ и $ O_2$ в отсутствие щели $S_2$ создают в плоскости П изображение щели $S_1$, и это изображение рассматривается в микроскоп М. Таким образом, нашу установку можно рассматривать как оптический инструмент, предназначенный для получения изображения предмета. При этом коллиматор (щель $S_1$ и объектив $O_1$) является моделью далёкого предмета, а объектив $O_2$ и микроскоп М составляют зрительную трубу, наведённую на этот предмет.
Щель $S_2$, установленная непосредственно перед объективом $O_2$, позволяет изменять эффективный размер объектива и, следовательно, разрешающую способность оптической системы.

\begin{figure}[h!]
	\centering
	\includegraphics[width=0.8\linewidth]{d.pdf}
	\caption{Схема установки для исследования разрешающей
		способности оптического инструмента}
	\label{labG}
\end{figure}

Поместим вместо щели $S_1$ экран Э с двумя узкими щелями, расстояние между которыми равно $d$ (рис. \ref{labG}). Тогда расстояние $l$ между изображениями щелей в плоскости П равно
\begin{equation}
	l = \varphi f_2 = d \dfrac{f_2}{f_1},
\end{equation}
а ширина каждого изображения
\begin{equation}
	\delta x \approx \dfrac{\lambda}{b} f_2
\end{equation}
определяется дифракцией света на щели $S_2$. Когда полуширина дифракционного изображения превышает расстояние между изображениями, то по виду дифракционной картины трудно определить, представляет собой источник двойную или одиночную щель.

Условия, при которых ещё можно различить, имеем мы дело с одной или двумя щелями, для разных наблюдателей различны. Для того чтобы исключить связанный с этим произвол, пользуются обычно критерием Рэлея, который приблизительно соответствует возможностям визуального наблюдения: изображения считаются различимыми, когда максимум одного дифракционного пятна совпадает с минимумом другого, а в условиях нашей задачи --- когда полуширина дифракционного изображения $\delta x$ совпадает с расстоянием $l$ между изображениями отдельных щелей:
\begin{equation}
	\delta x \sim l \to \dfrac{\lambda}{b} \sim \dfrac{d}{f_1}.
\end{equation}

\textbf{Ход работы и обработка результатов.}\\

Подберем ширину $b_0$ щели $S_2$ так, чтобы изображения обеих щелей почти сливались,
но всё-таки ещё воспринимались раздельно: $b_0$ = $0.146 \pm 0.02$ мм.

Для измерения размеров двойной щели, поставим её непосредственно перед микроскопом и измерим с помощью микрометрического винта поперечных салазок микроскопа
расстояние $d$ между щелями и ширину каждой щели $D$:

$d$ = $0.71 \pm 0.01$ мм

$D_1$ = $0.19 \pm 0.01$ мм

$D_2$ = $0.3 \pm 0.01$ мм


Проверим разрешающую способность по критерию Рэлея, сравнивая измеренную ширину $b_0$ щели $S_2$ с расчётом по формуле (14):

$b = 0.23 \pm 0.01$ мм

	\textbf{Обсуждение результатов и выводы: }\\
	
Мы изучили два основных типа дифракции: Френеля и Фраунгофера при разных размерах щели и провели качественные наблюдения этих явлений, а также экспериментально проверили справедливость теоретических формул.

	
\end{document}